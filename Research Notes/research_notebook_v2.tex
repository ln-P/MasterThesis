\documentclass{article}
\usepackage[utf8]{inputenc}
\usepackage[margin=2cm]{geometry}
\usepackage{xcolor}

% Seting up notes
\usepackage[draft]{todonotes}
\presetkeys{todonotes}{color=blue!30}{}

% Removing Abstract name
\renewcommand{\abstractname}{\vspace{-\baselineskip}}

% Paragraphs have no indent, but a line skip instead
\usepackage[parfill]{parskip}

\title{\vspace{-2.5cm} \textbf{Research Project Notebook}}
\author{Wiktor Owczarz}
\begin{document}

\maketitle

\section{Problem Statement}

\subsection*{Rationale}

\textbf{\textit{Hypothesis:}} central bank researchers face higher \textbf{reputation risk}, compared to university researchers, when publishing banking competition studies. Economists employed by central banks are placed
under \textbf{higher public scrutiny}, hence their publications can have stronger \textbf{signalling effect}.\\

\textbf{\textit{As a result:}} central bank researchers are more prone to making \textbf{Type II Error} (failing to reject false null hypothesis). The cost of making \textbf{Type I Error} is high and might lead to financial instability.\\

\todo[inline, color=red!30]{Define reputation/signalling in context of economic theory}
\todo[inline, color=red!30]{Define role of a central bank (supervisor/regulator) + literature}


\subsection*{Objectives}
\textbf{\textit{Main Goal:}} Quantify if there are significant, measurable differences between banking competition papers published by the two groups of researchers. \\

\textbf{\textit{Thesis Purpose:}} Create \textbf{methodology} that can be applied to analyse and measure the differences in  competition papers.\\
\todo[inline, color=red!30]{Short: write methodology paper, define \textbf{heterogenity} and to measure it in a robust way}

\newline \noindent
\textbf{Possible \textit{"sources"} of heterogeneity:}
\begin{enumerate}
    \item \textbf{Meta Study:} When looking at the same market, using the same \textit{(similar?)} methodology two groups get either contrary or close but \textbf{different} (insignificant, smaller confidence, higher variability) outcomes.
    \item \textbf{Text Analysis:} Looking at the two groups we find significant \textbf{differences in analysed texts}. As a result, we would be able to classify papers to either group based on the text structure.
        \begin{enumerate}
            \item Apply Latent Semantic Analysis to find \textbf{proximity} between different papers based on \textbf{bag of words approach}.
             \item
             \item
        \end{enumerate}
\end{enumerate}

\newpage

\section{Methodology}

Objective is to \textbf{create number of tools} that will allow us to analyse and compare competition texts and quantify if any differences. \\
\begin{enumerate}
    \item \textbf{Initial Analysis}
        \begin{enumerate}
            \item Create corpus from the collected articles
            \item Transform data into text vectors
            \item Use tf-idf to weight the words
            \item Use SVDTruncated to reduce dimensions
            \item Run simple KMeans
        \end{enumerate}
        \todo[inline]{Very simplistic approach and does not fully capture the propose of the task}

    \item \textbf{Features engineering:}
        \begin{enumerate}
            \item word density: avg. length of words
            \item punctuation cound
            \item part of speech distributions
            \item lexical diversity
            \item \textbf{Uncertainty measure}: \% share of the words that introduce ambiguity in the text
        \end{enumerate}
         \todo[inline]{Features add further complexity but ignore the data, calculations}

\end{enumerate}
\newpage

\section*{References: (work-in-progress)}

\begin{enumerate}
    \item Stanley et. al. \textit{Meta-analysis of economics research reporting guidelines}
    \item A. P. Field, R. Gillett \textit{How to do a meta-analysis}, British Journal of Mathematical and Statistical Psychology
\end{enumerate}
\end{document}
