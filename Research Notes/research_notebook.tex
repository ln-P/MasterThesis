\documentclass{article}
\usepackage[utf8]{inputenc}
\usepackage[margin=2.5cm]{geometry}
\usepackage{xcolor}

% Seting up notes
\usepackage[draft]{todonotes}
\presetkeys{todonotes}{color=blue!30}{}

% Removing Abstract name
\renewcommand{\abstractname}{\vspace{-\baselineskip}}

% Paragraphs have no indent, but a line skip instead
\usepackage[parfill]{parskip}

\title{\vspace{-2.5cm} \textbf{Research Project Notebook}}
\author{Wiktor Owczarz}
\begin{document}

\maketitle

\begin{minipage}{0.8\textwidth}
\begin{abstract}
   The following document is an outline and notebook concerning a review study that aims to quantify between group differences in competition studies written by two groups of economists: Central Bank  vs. University.
\end{abstract}
\end{minipage}

\section{Problem Statement}

\subsection*{Rationale}
\todo[inline]{Describe the rationale for the review in the context of what is already known. (literature review + Hypothesis)}

\textbf{\textit{Hypothesis:}} Central Bank Researchers face higher \textbf{reputation risk}, compared to university researchers, when publishing banking compatition studies. Economists employed by central banks are placed under \textbf{higher public scrutiny}, hence their publications can have stronger \textbf{signalling effect}.

\todo[inline, color=red!30]{Define reputation/signalling in context of economic theory}

\subsection*{Objective}
\todo[inline]{What we do and how to do it correctly}
\textbf{\textit{Goal:}} We aim to quantify if there are significant, measurable differences between banking competition papers published by two groups of researchers. \\

\newline \noindent
\textbf{Possible \textit{"sources"} of heterogeneity:}
\begin{enumerate}
    \item \textbf{Meta Study:} When looking at the same market, using the same \textit{(similar?)} methodology two groups get either contrary or close but \textbf{different} (insignificant, smaller confidence, higher variability) outcomes.
    \item \textbf{Semantic Analysis:} Looking at the two groups we find significant semantic differences. As a result, we would be able to classify paper to either group based on the words they use.
        \begin{enumerate}
            \item First idea: Latent Semantic Analysis to find \textbf{proximity} between different papers.
             \item If an \textit{algorithm} would be able to easly distinguish between two groups, then
        \end{enumerate}
\end{enumerate}

\todo[inline, color=red!30]{Goal is to define \textbf{heterogenity} and to measure it in a robust way}

\newpage

\section{Methodology}

\subsection{Meta-Study:}
Generally accepted structure required for the meta-study:

\subsubsection{Save study protocol \textit{before} actual study}

The \textit{before} protocol \textbf{has to include}:
\begin{itemize}
    \item Problem statement
    \item Literature search strategy

        \begin{itemize}
            \item Which databases
            \item Key words/definition of the search strategy
            \item Publication restrictions (e.g. language)
        \end{itemize}

    \item Exclusion/inclusion criteria. Decide which of the collected papers to include/exclude.

        \begin{itemize}
            \item Clear distinction between central bank and who university economist.
            \todo[inline]{After the selection process is done create PRISMA Flow diagram to summarise the articles exclusion/inclusion process.}
        \end{itemize}

\end{itemize}

\subsubsection{Calculate the size effects}
\textbf{Def.} Effect size is usually a standardised measure of the magnitude of observed effect (Clark-Carter, 2003; Field, 2005c).

\subsubsection{Do the basic meta-analysis}
Estimate effects in the population by combining
the effect sizes from a variety of articles. The estimate is a weighted mean of the effect sizes. \\

\todo[inline]{Here ideal outcome would be to find clear \textbf{heterogeneity} in the effect sizes between two groups.}



\subsection{Language Processing / Semantic Analysis}
The initial idea is to run Latent Semantic Analysis and by looking at word frequency in each document try to classify papers based on their comparability/similarity between each other.



\section{Notes:}

\begin{itemize}
    \item Affiliation based on: https://ideas.repec.org
\end{itemize}


\section{Problems}

\subsection*{The Effect size}

According to Stanley et al. in \textit{“Reporting Guidelines for Meta-Regression Analyses in Economics”}, in order to conduct successful meta-study:
\begin{itemize}
    \item A precise definition of how effects are measured (the ‘effect size’), accompanied by any relevant formulas.
    \item An explicit description about how measured effects are comparable, including any methods used to standardise or convert them to a common metric.
\end{itemize}

\newline \noindent
As I already pointed out we face significant difficulty when it comes to the second point. Also we cannot just look at the sign and/or significance since the main purpose of the meta-study is to aggregate multiple papers to average out the bias that might come from single study and therefore find the \textit{"true" outcome}.   \\

\newline \noindent
However, maybe we could use \textbf{Cohen’s d}: $\frac{\mu_1 - \mu_2}{\sigma}$, where two means would be papers classified to central bank or university and the variance can be a pooled estimate based on both groups individual size and variances (for me it looks very similar to pooled t-static for unequal variances). We would have to do the comparison for individual models separately. \\

\newline \noindent
Although, just looking at Panzar-Rosse I can see problem since different papers use \textbf{different estimation methods}:
\begin{enumerate}
    \item Revenue equation: $log(TR)$ with $log(TA)$ as a control variable
    \item $log(TR/TA)$  without $log(TA)$ as a control variable
    \item Price equation: $log(TR/TA)$ with $log(TA)$ as a control variable
\end{enumerate}

\newpage

\section*{References: (work-in-progress)}

\begin{enumerate}
    \item Stanley et. al. \textit{Meta-analysis of economics research reporting guidelines}
    \item A. P. Field, R. Gillett \textit{How to do a meta-analysis}, British Journal of Mathematical and Statistical Psychology
\end{enumerate}
\end{document}
